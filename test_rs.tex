\documentclass[11pt, oneside]{article}   	% use "amsart" instead of "article" for AMSLaTeX format
\usepackage{geometry}                		% See geometry.pdf to learn the layout options. There are lots.
\geometry{letterpaper}                   		% ... or a4paper or a5paper or ... 
%\geometry{landscape}                		% Activate for rotated page geometry
%\usepackage[parfill]{parskip}    		% Activate to begin paragraphs with an empty line rather than an indent
\usepackage{graphicx}				% Use pdf, png, jpg, or eps§ with pdflatex; use eps in DVI mode
								% TeX will automatically convert eps --> pdf in pdflatex		
\usepackage{amssymb}
\usepackage{tabularx}				% To automatically wrap long column contents

%SetFonts

%SetFonts


\title{Test}
\author{Rahul Subramanian}
%\date{}							% Activate to display a given date or no date

\begin{document}
\maketitle
%\section{}
%\subsection{}
\newpage

Q1 

\begin{center}
 \begin{tabular}{||c c c c||} 
 \hline
 Col1 & Col2 & Col2 & Col3 \\ [0.5ex] 
 \hline\hline
 1 & 6 & 87837 & 787 \\ 
 \hline
 2 & 7 & 78 & 5415 \\
 \hline
 3 & 545 & 778 & 7507 \\
 \hline
 4 & 545 & 18744 & 7560 \\
 \hline
 5 & 88 & 788 & 6344 \\ [1ex] 
 \hline
\end{tabular}
\end{center}

\begin{center}
 \begin{tabular}{||c c c c||} 
 \hline
Skill & Level & Description & Support \\ [0.5ex] 
 \hline\hline
 PID Control & 4 & --- & --- \\
 \hline
 Model Predictive Control & 2 & --- & I need to put in the time on this one, I don't need external support. \\
 \hline
 Optimal Control & 2 & --- & I need to learn more about dynamic programming. \\
 \hline
 Dynamic systems modeling & 4 & The only caveat here is, ODEs are a lot friendlier & None, I think PDEs are within reach \\
 \hline
 Microcontrolers & 3 & I put my code on them, though don't write drivers or work with the OS & Firmware engineer \\
 \hline
 Scripting languages & 4 & Matlab & --- \\
 \hline
 C/C++ & 2 & --- & Firmware engineer, but this is on my list of things I need to get very good at. \\
 \hline
 Software version control & 2 & I use it on the reg, but if I run into an issue I call the Git expert & --- \\
 \hline
 Experimental Design & 1 & I just googled this & --- \\
 \hline
 Instrumentation & 2 & I can do this no problem, though it's just generally the systems/test engineer that has handled this in my experience. For example, I generally have taken care of selecting sensors and deciding their interfacing with the micro. & None, I can do this as required, I'll need to read up on different kinds of thermocouples to be totally useful. \\
\hline
Data acquisition (NI or other DAQ) & 2 & Same answer as 'Instrumentation' & --- \\
\hline
Project management & 3 & Part of my responsibility is to scope out my work, and plan how long it could possibly take. As I have spent time in thermal teams, I need to figure this out for myself. & --- \\
\hline
Written communication & 3 & I hope you like this document. & --- \\
\hline
Presentation & 3 & --- & --- \\
\hline
Heat transfer & 3 & I have a working knowledge of heat transfer. I don't calculate convective heat transfer coefficients or design HXs. But if I know the U value as a function of inner/outer mdots, I can create a simulation no problem. & A heat transfer engineer that understands convection well. \\
\hline
Thermodynamics & 3 & I'm very comfortable with P-h diagrams for example, and we should add another dimension (mdot) to it as it's quite incomplete to contemplate operating points without fixing evaporator power flux. As with heat transfer though, my working knowledge is not deep - I won't be able to derive the equation of work done for polytropic compression, though I find that equation quite useful for the compressor & Thermodynamics Wrangler \\
\hline
Fluid mechanics & 2 & --- & --- \\
\hline

\end{tabular}
\end{center}

\begin{tabularx}{\textwidth}{l X X l}
\hline
No. & I'm very comfortable with P-h diagrams for example, and we should add another dimension (mdot) to it as it's quite incomplete to contemplate operating points without fixing evaporator power flux. As with heat transfer though, my working knowledge is not deep - I won't be able to
    & These are very long too... 
    & 0\\
\hline
...
\end{tabularx}

\end{document}  