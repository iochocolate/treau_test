\documentclass[11pt, oneside]{article}   	% use "amsart" instead of "article" for AMSLaTeX format
\usepackage{geometry}                		% See geometry.pdf to learn the layout options. There are lots.
\geometry{letterpaper}                   		% ... or a4paper or a5paper or ... 
%\geometry{landscape}                		% Activate for rotated page geometry
%\usepackage[parfill]{parskip}    		% Activate to begin paragraphs with an empty line rather than an indent
\usepackage{graphicx}				% Use pdf, png, jpg, or eps§ with pdflatex; use eps in DVI mode
								% TeX will automatically convert eps --> pdf in pdflatex		
\usepackage{amssymb}
\usepackage{tabularx}				% To automatically wrap long column contents

\title{\textbf{Take Home Test}}
\author{Rahul Subramanian}
%\date{}							% Activate to display a given date or no date

\begin{document}
\maketitle

\newpage

\section{Question}
1. Assess yourself, from 1 to 4, in the following areas or skills. We are not looking for a candidate to have experience with all, or even a majority, of these areas. This question is much less about the magnitude of the numbers and more about their relation – where do you think your strengths are?
\newline
\newline
1 = Very little or no experience
\newline
2 = Some experience, but not yet at a level of full competence and confidence
\newline
3 = A lot of experience, generally able to do this work at a professional level
\newline
4 = Very experienced, typically the expert on my team or among peers
\newline
\begin{itemize}
    \item PID control
    \item Model predictive control
    \item Optimal control methodologies
    \item Dynamic systems modeling
    \item Microcontrolers
    \item Scripting languages
    \item C/C++
    \item Software version control
    \item Experimental design
    \item Instrumentation
    \item Data acquisition (NI or other DAQ)
    \item Project management
    \item Written communication
    \item Presentation
    \item Heat transfer
    \item Thermodynamics
    \item Fluid mechanics
\end{itemize}

\subsection *{Answer}
\begin{center}
\begin{tabularx}{\textwidth}{X l X X}
\hline
	Skill &
	Level & 
	Description & 
	Support 
	\\ [0.5ex] 
\hline\hline
	PID Control & 
	4 & 
	--- & 
	--- \\
\hline
	Model Predictive Control & 
	2 &
	--- & 
	I need to put in the time on this one, external support not required. \\
\hline
	Optimal Control & 
	2 & 
	I need to learn more about dynamic programming for dynamic optimization. But I do kalman filters and LQR & 
	--- \\
\hline
	Dynamic systems modeling & 
	4 & 
	ODEs are friendly, PDEs are within reach & 
	A thermal science expert, to bake what they know into the dynamics \\
\hline
	Microcontrollers & 
	3 & 
	--- & 
	Firmware engineer to own OS/drivers \\
\hline
	Scripting languages & 
	4 & 
	Matlab & 
	--- \\
\hline
	C/C++ & 
	2 & 
	--- & 
	Firmware engineer to fill the gap, though I can (and desperately want to) come up to speed with this. Expect me to 	become useful on this within a quarter or two or joining. I have in the past done all my work including real-time code in Simulink \\
\hline
	Software version control & 
	2 & 
	I use it regularly & 
	Git Wrangler \\
\hline
	Experimental Design & 1 & I just googled this & --- \\
\hline
	Instrumentation & 
	2 & 
	It's the systems/test engineer that has handled this in my experience, but I can do this. Related skill - I generally pick sensors and 		decide (with electrical engineer) interfacing with the micro. & 
	I will need to read the manual on thermocouples, anemometers, etc \\
\hline
	Data acquisition (NI or other DAQ) & 
	2 & 
	Same answer as 'Instrumentation' & 
	--- \\
\hline
\end{tabularx}

\begin{tabularx}{\textwidth}{X l X X}
\hline
	Project management & 
	3 & 
	Part of my responsibility is to scope out my work. As I have spent time in thermal teams, there's no one to guide me on this. & 
	--- \\
\hline
	Written communication & 
	3 & 
	I hope you like this document. & 
	--- \\
\hline
	Presentation & 
	3 & 
	--- & 
	--- \\
\hline
	Heat transfer & 
	3 & 
	I have a working knowledge of heat transfer. I don't calculate convective heat transfer coefficients or design HXs. But if I know the U value as a function of inner/outer mdots, I can create a simulation no problem. 		& 
	Thermal Engineer \\
\hline
	Thermodynamics & 
	3 & 
	By example - I am very comfortable with P-h diagrams, and we should add mdot as a 3rd dimension to 		contemplate operating points for a constant evaporator power flux to make it complete. I cannot derive the 	equation of work done for polytropic compression, though I find that equation quite useful for the 			compressor & 
	Thermal Engineer \\
\hline
Fluid mechanics & 2 & --- & Thermal Engineer \\
\hline

\end{tabularx}
\end{center}


\end{document}  