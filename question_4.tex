\section{Question}
Tell us about the time you decided you wanted to follow your current career path
(engineer/researcher/etc.).

\subsection *{Answer}
I grew up fascinated by cars. By the late 90s, India's experiment with economic liberalization was paying off. The car market was oxygenated by the influx of East Asian and German brands for the first time, which I followed fervently. I had a favorite motorcycle designer, favorite automotive magazine journalists, got the whole family hooked on Formula 1, and googled the Art Center of Design in Pasadena as soon as I got an internet connection.

I pursued mechanical engineering with the sole intent of joining the car industry. By my early 20s, my dogmatic obsession with the market faded, but I persisted with working in clean transportation. Clemson had a collaboration with BMW, which is why I went there (got a fellowship too, free money is cool). We were working on making a hybrid out of a BMW 1-series. I was the guy you went to for your Matlab/Simulink homework. We needed someone to work on the controller for the vehicle, that became me by default. That is how I got into control engineering. A few hours after I flew out to join Daimler, the car ran for the first time.

I joined Daimler to work on their powertrain harware-in-loop simulation team, but a budget cut axed it. They were working on project Supertruck with the DOE, and I knew the same toolchain from the BMW project. By a stroke of serendipity, I landed on a thermal control project.

Faraday Future and Zoox followed. Electric cars have high-fidelity thermal networks. For example, consider 2 condensers, 2 evaps, 2 EXVs, 2 pumps, 3-way valves, 4-way valves, multiple fans in concert to cool, heat pump, heat scavenge etc with variable position/speed actuators. Its a great problem to work on. If I was a company founder (which I am not sure I am \textit{right now}), I would work on building a residential central HVAC unit; a vantage point of controls gives me a different perspective. For example - using one PT sensor to control compressor input superheat \textit{instead of after each evaporator TXV-style}, with independent heat vectoring for a dual evaporator system saves energy by dropping compressor input superheat and therefore the pressure ratio. That is not available on any car last I checked, I am not sure of stationary refrigeration systems. But I want to make that the norm and hence my proclivity for thermal/control engineering.

If Silicon Valley created hardware companies that comported with its environmental values, thermal startups would not be uncommon. Working on this product would be among the better things at present time that I could do to save energy, and hence my application with Treau.